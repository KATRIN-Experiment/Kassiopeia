% Introduction for the Kassiopeia Guide

\chapter{Introduction}\label{sec:introduction}

\textsc{Kassiopeia} is the primary simulation package for the KATRIN Experiment.  It is written primarily in C++, and is comprised of modules for particle creation, particle trajectory calculation in electro-magnetic fields, particle detection and data acquisition. A variety of different particle generators (see Section~\ref{KPAGEmain}) and different tracking methods are available. Physical processes like synchrotron radiation, scattering can be taken into account (see Section~\ref{sec:KTrack}). The particle detection module includes backscattering of electrons on the detector surface as well as a comprehensive number of physical phenomena of low energetic electrons in silicon (see Section~\ref{KESSMain}). DAQ is not available in this version of \textsc{Kassiopeia}. The user interface is via configuration files and output is written into \cernroot{} files.

The Monte Carlo simulations performed with \textsc{Kassiopeia} should be applicable for three purposes: 
\begin{enumerate}
\item Answering design questions for KATRIN. For instance, the influence of a gap in the wire electrode system on the background could be studied with \textsc{Kassiopeia}.
\item \textsc{Kassiopeia} will be used for near-time simulations, in the commissioning and data-taking phases of KATRIN. This will give the experimenters the ability to quickly check or preview of the measurements.
\item Finally, with the help of Monte Carlo simulations with \textsc{Kassiopeia}, the systematic uncertainties for a neutrino-mass measurement can be estimated.
\end{enumerate}


